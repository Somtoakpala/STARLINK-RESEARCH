\chapter{INTRODUCTION}
\section{CONTEXT OF STUDY}
History has it that in August 1962, J.C.R. Licklider of MIT wrote a series of memoranda outlining his "Galactic Network" concept, which was the first documented account of the social connections that networking could facilitate. \cite{leinerbrief}. In the course of the Cold world war, the internet taken as a government project, to establish communication in the military that can not be damaged by nuclear bomb \cite{abbate2000inventing}.After the war, modern day internet was born. On April 30, 1995, the America government halted operations of the old NSFNET backbone and handed over the infrastructure to private companies \cite{abbate2000inventing}.\\
	According to \cite{mack2020history} the mode of connection to this early internet was through modems connected over telephone line charging in per use basis at a speed 33.6 and 56 kbps. This means that it would take 1 day, 18 hours and 36 minutes to download a 1 gigabyte file. As the need for faster internet become more glaring, broadband became available to the public with better speed and constant connection unlike modems which require dial-up before connection\\.
	\cite{moeyaert2011network} explains that there are roughly types of broadband connection technologies currently in use. Among which are digital subscriber line (DSL), cable modem, fiber, wireless, satellite, and broadband over powerlines.\\
	Satellite broadband which is our focus, was first consider for internet usage in 1986, by Viasat a company start by Mark Dankberg, Steve Hart and Mark Miller (source https://news.viasat.com/blog/scn/satellite-communications-a-brief-history-from-sputnik-to-viasat-3). With the technology developing rapidly, companies ventured into using GEO but it comes with disadvantages yielding to the exploration of LEO.\\
	SpaceX Star-link a satellite internet service provider licensed for operation on May 2022 in Nigeria (NCC 2023), has raised questions concerning her reliability to deliver better internet performance for Nigerians where the need for fast internet is rampant as technology develops rapidly.
	The regular internet service providers in Nigeria, employs GEO technology where ground stations communicates with the satellite orbiting 35838km from the earth surface. This approach of internet connection is said to be characterized by slow speed and it has high latency and longer routing strategy.
	In a bid to providing a solution, a more recent approach to internet connectivity, that is been employed by SpaceX Starlink has been developed- LEO (Low Earth Orbit Satellite).
	Satellites here are lunched to an orbit 523km away from the earth. With this, Star-link is excepted to have better performance in terms of Latency, Routing Strategy and throughput as the distance is shorter and direct communication with the Dishy is made possible.\\
	For star-link broadband to function, it require the communication between a white surface device called Dishy Mcflatface (Dishy for short)which is placed in an obstruction free location and star-link satellite orbiting at a speed of 27000km/hr at the LEO.
	The dishy consist of various layers but most importantly, 1280 antennas which are arranged in a honeycomb pattern. Communicate with the fast moving satellite is made possible by including phased array technology which helps steer the beams from the antennas without physically moving it. Various versions of dishy has been launched with a visible difference in flatness and size thereby making starlink suitable for mobility\\
	The first lunched star-link requires the user to be about 50km from ground stations. This  is necessary because dishy needs to communicate with the ground-station of its required data which is then related to the star-link satellite before the later transmits to the dishy; it is termed bentpipe architecture. Version 2 Satellite launched in April, 2023 is equipped with LASER communication to enable the satellite communicate within themselves on the orbit, thereby limiting the need for ground stations.

\section{Motivation for Study}
 With SpaceX star-link operation in Nigeria, the cost for purchase goes for about 290k to 350k. For a average Nigeria, the price is relatively high for a product with little or no empirical analysis done to compare it's performance with already established broadband provides that cost less. Hence the question should I spend that much?\\
Quiet a number of people have tried answering by comparing star-link performance in high internet speed demanding platforms such as gaming and streaming. It is important to state that their review from such test can not be free from major errors as the performance is not in it's totality dependent on the broadband as there could external influences too. Hence this research to empirically do a testing that follows proper scientific procedures.

\section{Aim of study}
The primary purpose for this research is to compare the performance of SpaceX star-link with Mikrotik Broadband. This is done first by testing the performance of both broadband separately before comparing
\section{SCOPE}

\section{METHODOLOGY}

